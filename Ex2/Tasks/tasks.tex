\documentclass[12pt,letterpaper]{article}
\usepackage{fullpage}
\usepackage[top=2cm, bottom=4.5cm, left=2.5cm, right=2.5cm]{geometry}
\usepackage{amsmath,amsthm,amsfonts,amssymb,amscd}
\usepackage{lastpage}
\usepackage{enumerate}
\usepackage{fancyhdr}
\usepackage{mathrsfs}
\usepackage{xcolor}
\usepackage{graphicx}
\usepackage{listings}
\usepackage{hyperref}

\hypersetup{%
  colorlinks=true,
  linkcolor=blue,
  linkbordercolor={0 0 1}
}
 
\renewcommand\lstlistingname{Algorithm}
\renewcommand\lstlistlistingname{Algorithms}
\def\lstlistingautorefname{Alg.}

\lstdefinestyle{Python}{
    language        = Python,
    frame           = lines, 
    basicstyle      = \footnotesize,
    keywordstyle    = \color{blue},
    stringstyle     = \color{green},
    commentstyle    = \color{red}\ttfamily
}

\setlength{\parindent}{0.0in}
\setlength{\parskip}{0.05in}

% Edit these as appropriate
\newcommand\course{Search Based Software Engineering}
\newcommand\hwnumber{2}                  % <-- homework number
\newcommand\NetIDb{Lucas H\"ubner, 116232}           
\newcommand\NetIDa{Hans Lienhop, 114926}           

\pagestyle{fancyplain}
\headheight 35pt
\lhead{\NetIDa}
\lhead{\NetIDa\\\NetIDb}                 % <-- Comment this line out for problem sets (make sure you are person #1)
\chead{\textbf{\Large Exercise \hwnumber}}
\rhead{\course \\ \today}
\lfoot{}
\cfoot{}
\rfoot{\small\thepage}
\headsep 1.5em

\begin{document}

\section*{Task 1}
\textbf{a) What are $(1+1)$, $(1 + \lambda)$, $(1,\lambda)$ and $(\mu,\lambda)$?}\\
They describe instances of evolutionary strategies.\\

\textbf{b) Explain differences between them.}
\begin{enumerate}
\item[•] $(1+1)$: One parent produces one candidate solution and parent and child compete based on objective fitness for a position in the next generation.
\item[•] $(1 + \lambda)$: One parent produces $\lambda$ candidate solutions and parent and children compete based on objective fitness for a position in the next generation.
\item[•] $(1 , \lambda)$: One parent produces $\lambda$ candidate solutions and only the children compete based on objective fitness for a position in the next generation while the parent is disregarded.
\item[•] $(\mu , \lambda)$: The $\mu$ fittest parents from randomly selected $\lambda$ individuals from the population produce $\frac{\lambda}{\mu}$  candidate solutions each and these children form the new population.
\end{enumerate}

\section*{Task 2}
a) How does Line Recombination work? Explain in detail. \\

b)  How can Line Recombination be extended to get Intermediate Line Recombination? \\

c)  Implement Intermediate Line Recombination as a python function.

\section*{Task 3}
% Rest of the work...
a)  What is Fitness-Proportionate Selection (FPS) and how does it work? \\

b)  What is Stochastic Universal Sampling (SUS) and how does it work? \\

c)  Implement SUS as a python function.
\end{document}